\section*{Contexte g\'en\'eral}

Les ph\'enom\`enes cycloniques repr\'esentent des syst\`emes dynamiques complexes ayant un impact majeur sur les r\'egions tropicales et subtropicales. Leur compr\'ehension est essentielle tant pour des raisons scientifiques que pour des enjeux soci\'etaux li\'es \`a la pr\'evention des catastrophes naturelles. La mod\'elisation math\'ematique de tels ph\'enom\`enes permet de traduire les m\'ecanismes physiques en \'equations diff\'erentielles, d\'eriv\'ees des lois fondamentales de la m\'ecanique des fluides.

Cette approche permet d'\'etudier, de mani\`ere th\'eorique ou num\'erique, des aspects tels que la formation, l'\'evolution et la structure interne des cyclones. En particulier, les mod\`eles simplifi\'es bas\'es sur des hypoth\`eses de sym\'etrie ou d'\'equilibre permettent de mieux saisir les ph\'enom\`enes dominants sans recourir \`a des simulations num\'eriques lourdes.

\section*{Objectif du stage}

Ce stage de recherche, r\'ealis\'e dans le cadre du Master 1 \og Math\'ematiques et Applications \fg{}, a pour objectif l'\'etude d'un syst\`eme d'\'equations diff\'erentielles ordinaires (EDO) issu de la mod\'elisation des cyclones. Ce syst\`eme, bas\'e sur un mod\`ele en sym\'etrie de r\'evolution, est pr\'esent\'e dans le pr\'eprint d'Alain-Yves LeRoux intitul\'e \textit{Mod\'elisation des \'ecoulements dans leur environnement} (2014).

L'analyse porte notamment sur la compr\'ehension des hypoth\`eses physiques, la structure math\'ematique du mod\`ele, ainsi que sur les propri\'et\'es des solutions, en particulier celles associ\'ees \`a la formation du mur de l'\oe il d'un cyclone. Une attention particuli\`ere est port\'ee \`a la bifurcation de solutions au voisinage du bord ext\'erieur du syst\`eme, condition n\'ecessaire \`a l'apparition d'un comportement cyclonique structur\'e. Une mod\'elisation par EDP du champ de vitesses est \textit{\'egalement} envisag\'ee, et les deux approches sont compar\'ees par validation crois\'ee. Enfin, une \textbf{analyse asymptotique} rigoureuse est r\'ealis\'ee autour du mur de l'\oe il afin d'expliquer les singularit\'es du champ de vitesse pr\'edites par le mod\`ele.

\section*{D\'emarche adopt\'ee}

Le travail a d\'ebut\'e par une recherche bibliographique sur les \'equations de la dynamique atmosph\'erique, la force de Coriolis, et les mod\`eles d'\'ecoulements g\'eophysiques. Une lecture approfondie du pr\'eprint de LeRoux~\cite{leroux2014modelisation} a \'et\'e men\'ee, avec un accent sur :
\begin{itemize}
    \item la section~7.2, qui pr\'esente les \'equations du vent dans les r\'egions intertropicales ;
    \item la section~7.4.1, introduisant un mod\`ele cyclonique 2D simplifi\'e ;
    \item la section~7.4.2, concernant le mur de l'\oe il comme discontinuit\'e dans le champ de vitesses ;
    \item la section~7.4.3, d\'eveloppant un mod\`ele axisym\'etrique r\'egi par un syst\`eme d'EDO.
\end{itemize}

\`A partir de ces \'el\'ements, une phase de formalisation math\'ematique a permis d'analyser le comportement des solutions, les effets de la force de Coriolis et les conditions menant \`a la formation d'un cyclone. Des simulations num\'eriques ont \'et\'e r\'ealis\'ees pour les deux mod\`eles, suivies d'une phase de validation crois\'ee. L'analyse asymptotique a ensuite permis de caract\'eriser le comportement local du champ de vitesse au voisinage du mur de l'\oe il, en r\'ev\'elant les conditions limites internes compatibles avec le mod\`ele global.

\section*{Structure du m\'emoire}

Ce m\'emoire est organis\'e en cinq chapitres principaux :
\begin{itemize}
    \item Le \textbf{Chapitre 1} pose le cadre th\'eorique : \'equations d'Euler, mod\'elisation des \'ecoulements atmosph\'eriques, hypoth\`eses simplificatrices, et introduction \`a la force de Coriolis ;
    \item Le \textbf{Chapitre 2} expose la m\'ethodologie suivie pour l'\'etude et la simulation ;
    \item Le \textbf{Chapitre 3} pr\'esente les mod\`eles issus du pr\'eprint de LeRoux ;
    \item Le \textbf{Chapitre 4} discute les r\'esultats obtenus, les validations crois\'ees, les limites du mod\`ele, ainsi que les contributions de l'analyse asymptotique ;
    \item Le \textbf{Chapitre 5} propose une conclusion et des perspectives pour des travaux futurs.
\end{itemize}
