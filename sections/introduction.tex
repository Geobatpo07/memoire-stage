\section*{Contexte général}

Les phénomènes cycloniques représentent des systèmes dynamiques complexes ayant un impact majeur sur les régions tropicales et subtropicales. Leur compréhension est essentielle tant pour des raisons scientifiques que pour des enjeux sociétaux liés à la prévention des catastrophes naturelles. La modélisation mathématique de tels phénomènes permet de traduire les mécanismes physiques en équations différentielles, dérivées des lois fondamentales de la mécanique des fluides.

Cette approche permet d'étudier, de manière théorique ou numérique, des aspects tels que la formation, l'évolution et la structure interne des cyclones. En particulier, les modèles simplifiés basés sur des hypothèses de symétrie ou d’équilibre permettent de mieux saisir les phénomènes dominants sans recourir à des simulations numériques lourdes.

\section*{Objectif du stage}

Ce stage de recherche, réalisé dans le cadre du Master 1 « Mathématiques et Applications », a pour objectif l’étude d’un système d’équations différentielles ordinaires (EDO) issu de la modélisation des cyclones. Ce système, basé sur un modèle en symétrie de révolution, est présenté dans le préprint d’Alain-Yves LeRoux intitulé \textit{Modélisation des écoulements dans leur environnement} (2014).

L’analyse porte notamment sur la compréhension des hypothèses physiques, la structure mathématique du modèle, ainsi que sur les propriétés des solutions, en particulier celles associées à la formation du mur de l’œil d’un cyclone. Une attention particulière est portée à la bifurcation de solutions au voisinage du bord extérieur du système, condition nécessaire à l’apparition d’un comportement cyclonique structuré.

\section*{Démarche adoptée}

Le travail a débuté par une recherche bibliographique sur les équations de la dynamique atmosphérique, la force de Coriolis, et les modèles d’écoulements géophysiques. Une lecture approfondie du préprint de LeRoux \cite{leroux2014modelisation} a été menée, avec un accent sur :
\begin{itemize}
    \item la section 7.2, qui présente les équations du vent dans les régions intertropicales ;
    \item la section 7.4.1, introduisant un modèle cyclonique 2D simplifié ;
    \item la section 7.4.2, concernant le mur de l’œil comme discontinuité dans le champ de vitesses ;
    \item la section 7.4.3, développant un modèle axisymétrique régi par un système d’EDO.
\end{itemize}

À partir de ces éléments, une phase de formalisation mathématique a permis d’analyser le comportement des solutions, les effets de la force de Coriolis et les conditions menant à la formation d’un cyclone. Des pistes de visualisation numérique ont également été explorées en fin de parcours.

\section*{Structure du mémoire}

Ce mémoire est organisé en trois chapitres principaux :
\begin{itemize}
    \item Le \textbf{Chapitre 1} pose le cadre théorique : équations d’Euler, modélisation des écoulements atmosphériques, hypothèses simplificatrices, et introduction à la force de Coriolis.
    \item Le \textbf{Chapitre 2} expose en détail le modèle étudié, présente les équations issues du préprint, ainsi que leur traitement mathématique.
    \item Le \textbf{Chapitre 3} discute les résultats obtenus, les limites du modèle, et les perspectives d’amélioration ou d’extension.
\end{itemize}
