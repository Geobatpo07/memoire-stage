\chapter{R\'esultats et discussion}

Ce chapitre pr\'esente les principaux r\'esultats num\'eriques obtenus \`a partir des mod\`eles EDO et EDP. Les figures associ\'ees sont issues de simulations Python d\'evelopp\'ees dans le cadre de ce travail. Une validation croisée est ensuite men\'ee pour confronter les deux approches.

\section{Simulation EDO : structure radiale du cyclone}

Le syst\`eme d'\'equations diff\'erentielles ordinaires issu du mod\`ele axisym\'etrique a \'et\'e r\'esolu num\'eriquement par plusieurs m\'ethodes :
\begin{itemize}
    \item int\'egration classique de type Runge--Kutta (\texttt{solve\_ivp}),
    \item sch\'ema implicite de type BDF pour syst\`eme raide,
    \item m\'ethode r\'etrograde d'Euler explicite,
    \item mod\`ele r\'eduit via l'introduction du facteur g\'eostrophique $\theta(\xi)$.
\end{itemize}

Les profils des composantes radiale $a(\xi)$, tangente $b(\xi)$, du facteur $\theta(\xi)$ et de l'\'energie cin\'etique $E(\xi)$ ont \'et\'e trac\'es et compar\'es. Une forte coh\'erence entre les m\'ethodes a \'et\'e observ\'ee, sauf au voisinage de l'\oe il (l\`a o\`u $a \to 0$).

\section{Simulation EDP : champ de vitesse 2D}

Le mod\`ele EDP cartographique a \'et\'e simul\'e sur un domaine 2D via un sch\'ema explicite de type Lax--Friedrichs. Les champs de vitesse $(u(x,y), v(x,y))$ ont \'et\'e visualis\'es sous forme :
\begin{itemize}
    \item de champ vectoriel,
    \item de carte de la norme $\|v\|$,
    \item de trajectoires typiques des particules.
\end{itemize}

La structure tourbillonnaire attendue est clairement identifi\'ee. L'intensit\'e maximale de la vitesse est localis\'ee autour du mur de l'\oe il.

\section{Validation crois\'ee EDO / EDP}

Pour comparer les deux mod\`eles, un profil radial moyen de $\|v\|$ a \'et\'e extrait \`a partir du champ simul\'e en EDP. Ce profil a \'et\'e compar\'e aux valeurs th\'eoriques issues du mod\`ele EDO. Les courbes obtenues sont proches sur l'intervalle $[R_m, R_{\text{ext}}]$, validant la coh\'erence des deux approches.

\section{Analyse des \'ecarts entre m\'ethodes EDO}

Une comparaison des m\'ethodes de r\'esolution a \'et\'e r\'ealis\'ee en prenant la solution obtenue par Runge--Kutta comme r\'ef\'erence. Les \'ecarts relatifs suivants ont \'et\'e trac\'es :
\begin{itemize}
    \item $\Delta a(\xi)$, $\Delta b(\xi)$,
    \item $\Delta \theta(\xi)$,
    \item $\Delta E(\xi)$.
\end{itemize}

Ces graphes permettent de quantifier les erreurs introduites par les m\'ethodes plus simples (Euler, $\theta$-formulation). La m\'ethode \texttt{solve\_ivp} reste la plus robuste num\'eriquement.

\section{Visualisations graphiques}

Les figures int\'egr\'ees \`a ce travail montrent l'ad\'equation entre simulation et mod\`ele th\'eorique. La structure en spirale, l'intensit\'e au mur de l'\oe il, et l'organisation du champ vectoriel renforcent la validit\'e physique du mod\`ele adopt\'e.

\medskip

L'ensemble de ces r\'esultats valide la pertinence du mod\`ele d\'evelopp\'e et sugg\`ere plusieurs pistes d'approfondissement, notamment l'am\'elioration de la simulation proche du mur de l'\oe il ou l'int\'egration d'effets thermodynamiques plus complexes.
