\chapter{Th\'eorie}

\section{Mod\'elisation des \'ecoulements atmosph\'eriques}

La dynamique de l'atmosph\`ere est r\'egie par les lois de la m\'ecanique des fluides appliqu\'ees \`a un fluide compressible en rotation. Dans ce cadre, les \textit{\'equations de Navier--Stokes} ou d'\textit{Euler} sont accompagn\'ees de lois de conservation (masse, quantit\'e de mouvement, \'energie) et de conditions physiques (pression, temp\'erature).

Pour les ph\'enom\`enes m\'esoscopiques tels que les cyclones tropicaux, certaines hypoth\`eses de simplification sont admissibles :
\begin{itemize}
    \item Le fluide est trait\'e comme incompressible et sans viscosit\'e (fluide parfait) ;
    \item Le syst\`eme est consid\'er\'e dans un r\'ef\'erentiel en rotation avec la Terre ;
    \item La sph\`ere terrestre est projet\'ee sur une carte (projection de Mercator).
\end{itemize}

Ces hypoth\`eses conduisent \`a des mod\`eles d'\'ecoulements simplifi\'es, adapt\'es \`a une analyse math\'ematique et \`a une simulation num\'erique cibl\'ee.

\section{Effet de Coriolis et vents g\'eostrophiques}

La rotation terrestre induit une force fictive dite force de Coriolis, responsable de la d\'eviation des masses d'air. Dans l'h\'emisph\`ere nord, cette d\'eviation s'effectue vers la droite, et vers la gauche dans l'h\'emisph\`ere sud. Cette force modifie la dynamique des vents autour des zones de basse pression.

Elle s'exprime par :
\[
\vec{F}_C = -2m (\vec{\Omega} \times \vec{v}),
\]
o\`u $\vec{\Omega}$ est le vecteur de rotation terrestre et $\vec{v}$ la vitesse du fluide. Cette force est \`a l'origine de l'organisation circulaire des vents autour de l'\oe il des cyclones.

\section{Trois niveaux de mod\'elisation selon LeRoux (2014)}

Le pr\'eprint de A.-Y. LeRoux \cite{leroux2014modelisation} propose une hi\'erarchie de mod\`eles pour d\'ecrire les structures tourbillonnaires :

\subsection{Mod\`ele cartographique 2D (section 7.4.1)}

Il s'agit d'un syst\`eme d'\'EDP exprim\'e en projection cartographique, qui mod\'elise le champ de vitesses $(u,v)$ par rapport aux aliz\'es $(u^*,v^*)$ :
\[
\begin{cases}
\frac{\partial u}{\partial t} + u\frac{\partial u}{\partial x} + v\frac{\partial u}{\partial y} = (\xi^* - k)(u - u^*) + \omega_0 \sin\phi (v - v^*) \\
\frac{\partial v}{\partial t} + u\frac{\partial v}{\partial x} + v\frac{\partial v}{\partial y} = (\xi^* - k)(v - v^*) - \omega_0 \sin\phi (u - u^*)
\end{cases}
\]
Ce mod\`ele conserve les propri\'et\'es dynamiques de rotation, friction et for\c{c}age.

\subsection{Mod\'elisation du mur de l'\oe il (section 7.4.2)}

Une discontinuit\'e dans le champ de vitesses peut appara\^itre et correspondre au mur de l'\oe il. L'utilisation de la fonction de Heaviside permet d'exprimer math\'ematiquement ce saut. Le r\'esultat cl\'e est que le vecteur saut $(\Delta u, \Delta v)$ est tangent \`a la courbe $S(x - u^*t, y - v^*t) = 0$, front de discontinuit\'e.

\subsection{Mod\`ele en sym\'etrie de r\'evolution (section 7.4.3)}

Ce mod\`ele r\'eduit l'\'etude \`a deux fonctions scalaires $a(\xi)$ (radial) et $b(\xi)$ (tangent) :
\[
\begin{pmatrix}
U \\
V
\end{pmatrix}
= a(\xi)
\begin{pmatrix}
X \\
Y
\end{pmatrix}
+ b(\xi)
\begin{pmatrix}
- Y \\
X
\end{pmatrix}
\]
Ces fonctions v\'erifient un syst\`eme d'\'EDO non lin\'eaire. L'introduction du facteur g\'eostrophique $\theta(\xi) = -b(\xi)/a(\xi)$ permet de r\'e\'ecrire le syst\`eme sous forme r\'eduite et d'exhiber une bifurcation \`a l'entr\'ee du domaine.

Ce mod\`ele constitue le point de d\'epart des simulations pr\'esent\'ees, et de l'analyse asymptotique men\'ee au voisinage de $\xi_m$ o\`u $a(\xi) \to 0$, lieu critique de formation du mur de l'\oe il.

\medskip

Ce chapitre pose ainsi les fondations physiques et math\'ematiques du mod\`ele id\'ealis\'e de cyclone utilis\'e dans la suite du m\'emoire.
